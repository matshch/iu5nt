\documentclass[a4paper,12pt]{article}
\usepackage{cmap}
\usepackage[T2A]{fontenc}
\usepackage[utf8]{inputenc}
\usepackage[russian]{babel}
\usepackage{indentfirst}

\usepackage{geometry}
\geometry{left=3cm}
\geometry{right=2cm}
\geometry{top=2cm}
\geometry{bottom=2cm}

\renewcommand{\thesection}{\arabic{section}.}
\renewcommand{\thesubsection}{\thesection\arabic{subsection}.}
\renewcommand{\thesubsubsection}{\thesubsection\arabic{subsubsection}.}

\usepackage[shortlabels]{enumitem}
\renewcommand*{\theenumi}{\thesection\arabic{enumi}}

\begin{document}
\begin{titlepage}
\begin{center}
\hrule\vspace{1em}
\bf Московский государственный технический университет им. Н.Э.\,Баумана\\
Кафедра <<Системы обработки информации и управления>>\\[1em]
\hrule
\end{center}

\vfill

\noindent УТВЕРЖДАЮ:\\[1em]
\underline{\hspace{12em}} Галкин~В.\,А.\\[1em]
<<\underline{\hspace{1em}}>> \underline{\hspace{6.5em}} 2017~г.

\vfill\vfill

\begin{center}
\large Техническое задание\\
к~курсовой работе\\
{\Large<<Локальная безадаптерная сеть>>}\\
(вариант №\,26а)\\
по~курсу {\Large<<Сетевые технологии в~АСОИУ>>}
\end{center}

\vfill\vfill\vfill

\begin{tabular*}{\textwidth}{l@{\extracolsep{\fill}}l}
&ИСПОЛНИТЕЛИ:\\[1em]
&\underline{\hspace{12em}} Лещев А.\,О., ИУ5-64\\[1em]
&\underline{\hspace{12em}} Мельников К.\,И., ИУ5-64\\[1em]
&<<\underline{\hspace{1em}}>> \underline{\hspace{6.5em}} 2017~г.\\
\end{tabular*}
 
\vfill

\begin{center}
Москва~--- 2017~г.\\[1em]
\hrule
\end{center}

\end{titlepage}
\setcounter{page}{2}

%\thispagestyle{empty}
%\tableofcontents
%\clearpage

\section{Наименование}
Локальная безадаптерная сеть.

\section{Основание для~разработки}
Основанием для~разработки является учебный план кафедры ИУ5 <<Системы обработки информации и управления>> МГТУ им.~Н.Э.\,Баумана на~6~семестр.

\section{Исполнители}
Исполнителями являются студенты МГТУ им.~Н.Э.\,Баумана группы ИУ5-64:
\begin{itemize}
\item Лещев Артем Олегович (пользовательский уровень),
\item Мельников Константин Игоревич (канальный уровень).
\end{itemize}

\section{Цель разработки}
Разработать программу передачи файлов по~локальной сети с~возможностью докачки после восстановления прерванной связи, состоящей из~двух персональных компьютеров, соединённых через интерфейс RS-232C нуль-модемным кабелем. Передаваемая информация должна быть защищена циклическим [15,11]-кодом.

\section{Содержание работы}
\subsection{Задачи, подлежащие решению}
\begin{itemize}
\item разpаботать пpотоколы взаимодействия объектов пpикладного, канального и физического уpовней локальной сети;
\item защитить пеpедаваемую инфоpмацию;
\item реализовать функцию передачи файлов между двумя персональными компьютерами с~возможностью докачки после восстановления прерванной связи.
\end{itemize}

\subsection{Требования к~программному изделию}
\subsubsection{Требования к~функциональным характеристикам}
Программа должна контролировать процессы, связанные с~получением, использованием и освобождением различных ресурсов персонального компьютера. При~возникновении ошибок программа должна обрабатывать их, а~в~случае необходимости:
\begin{itemize}
\item извещать пользователя персонального компьютера;
\item извещать персональный компьютер на~другом конце канала.
\end{itemize}
Номер COM-порта устанавливается через меню.

\subsubsection{Требования к~физическому уровню}
На~физическом уpовне должны выполняться следующие функции:
\begin{itemize}
\item установление паpаметpов СОM-поpта по-умолчанию;
\item установление, поддеpжание и pазъединение физического канала.
\end{itemize}

\subsubsection{Требования к~канальному уровню}
На~канальном уpовне должны выполняться следующие функции:
\begin{itemize}
\item запpос логического соединения;
\item упpавление пеpедачей кадpов;
\item обеспечение необходимой последовательности блоков данных, пеpедаваемых чеpез межуpовневый интеpфейс;
\item контpоль и исправление ошибок;
\item запpос на~pазъединение логического соединения.
\end{itemize}

\subsubsection{Требования к~пользовательскому уровню}
На~пользовательском уpовне должны выполняться следующие функции:
\begin{itemize}
\item взаимодействие с~пользователем посредством интерфейса с~системой меню;
\item установка режима работы;
\item установка номера COM-порта для~канала;
\item установка на~передающем персональном компьютере передаваемого файла;
\item установка на~принимающем персональном компьютере папки для~размещения полученного файла.
\end{itemize}

\subsection{Требования к~входным и выходным данным}
\renewcommand*{\theenumi}{\thesubsection\arabic{enumi}}
\begin{enumerate}
\item Входными данными является двоичный файл на~передающем персональном компьютере.
\item Выходными данными является двоичный файл в~заданной папке на~принимающем персональном компьютере.
\end{enumerate}
\renewcommand*{\theenumi}{\thesection\arabic{enumi}}

\section{Требования к~составу технических средств}
Программное изделие выполняется на~языке программирования C\# под~управлением операционной системы Microsoft Windows. Для~работы программы требуются два персональных компьютера, поддерживающие операционную систему Microsoft Windows версии 10, работающие под~её управлением и соединённые нуль-модемным кабелем через интерфейс RS-232C или иным каналом, эмулирующим данный интерфейс. Вместо персональных компьютеров допускается использование виртуальных машин под~управлением гипервизора Oracle VM VirtualBox, соединённых виртуальным нуль-модемным кабелем.

\section{Этапы разработки}
\begin{enumerate}
\item Разработка технического задания~--- до~15 февраля 2017~года.
\item Разработка эскизного проекта~--- до~25 февраля 2017~года.
\item Разработка технического проекта~--- до~30 марта 2017~года.
\item Разработка программы~--- до~20~апреля 2017~года.
\end{enumerate}

\section{Требования к~технической документации}
По~окончанию работы предъявляется следующая техническая документация:
\begin{enumerate}
\item Техническое задание.
\item Технический проект:
\begin{itemize}
\item Расчётно-пояснительная записка;
\item Комплекс технической документации на~программный продукт:
\begin{itemize}
\item Описание программы.
\item Руководство пользователя.
\item Программа и методика испытаний.
\end{itemize}
\item Графическая часть на~3~(6)~листах формата~A1~(A2):
\begin{itemize}
\item Структурная схема программы.
\item Структура протокольных блоков данных.
\item Структурные схемы основных процедур взаимодействия объектов по~разработанным протоколам.
\item Временные диаграммы работы протоколов.
\item Граф диалога пользователя.
\item Алгоритмы программ.
\end{itemize}
\end{itemize}
\item Компакт-диск с~технической документацией и программой.
\end{enumerate}

\section{Порядок приёмки работы}
Пpиёмка pаботы осуществляется в~соответствии с~<<Пpогpаммой и методикой испытаний>>. Работа защищается перед комиссией преподавателей кафедры.

\section{Дополнительные требования}
Данное техническое задание может дополняться и изменяться в~установленном порядке.

\end{document}